\documentclass[usenames,dvipsnames]{beamer}

\RequirePackage[OT1]{fontenc}
\RequirePackage{amsthm,amsmath}
\RequirePackage[numbers]{natbib}
\resetcounteronoverlays{algocf}
\usepackage{appendixnumberbeamer}

\usepackage[utf8]{inputenc}
\usepackage[english]{babel}
\usepackage[ruled]{algorithm2e}
\usepackage{amsmath}
\usepackage{amsfonts}
\usepackage{amssymb}
\usepackage{changepage}
\usepackage{empheq}
\usepackage{mathrsfs}
\usepackage{tikz}
\usepackage{transparent}
\usepackage{xcolor}
\usepackage{xargs}

\usepackage{pifont}% http://ctan.org/pkg/pifont
\newcommand{\cmark}{\ding{51}}%
\newcommand{\xmark}{\ding{55}}%

\newcommand{\kd}[1]{\todo[color=red!25,size=\footnotesize]{ \textbf{KD:}  #1}}
\newcommand{\rd}[1]{\todo[color=blue!40,size=\footnotesize]{ \textbf{RD:}  #1}}
\newcommand{\fr}[1]{\todo[color=orange!40,size=\footnotesize]{ \textbf{FrR:}  #1}}
\newcommand{\fp}[1]{\todo[color=yellow!40,size=\footnotesize]{ \textbf{FrP:}  #1}}
\newcommandx{\admiss}[1][1=f]{\mathsf{A}_{#1}}
\newcommand{\arginf}{\mathrm{arginf}}
\newcommand{\argmin}{\mathrm{argmin}}
\newcommand{\argmax}{\mathrm{argmax}}
\newcommand{\argsup}{\mathrm{argsup}}
\newcommand{\as}{\mathrm{a.s.}}
%\newcommand{\b}[1]{{\mathbf{#1}}}
\newcommand{\ball}{\mathsf{B}}
\newcommandx{\binfty}[1][1=\alpha]{|b|_{\infty, #1}}
\newcommandx{\bmuf}[2][1=\mu, 2=\alpha]{ {b_{#1, #2}}}
\newcommandx{\bmufk}[2][1=\mu, 2=\alpha]{\hat{b}_{#1, #2, M}}
\newcommand{\bproof}{\textbf{Proof :}\quad}
\newcommand{\card}{\mathrm{card}}
\newcommand{\chunk}[3]{{#1}_{#2:#3}}
\newcommandx{\couple}[2][1=\PQ, 2=\PP]{(#1 || #2)}
\newcommand{\coupling}{\mathcal{C}}
\newcommand{\Cov}{\mathbb{C}\mathrm{ov}}
\newcommand{\cte}{\kappa}
\newcommandx{\ctemono}[1][1=\alpha]{L_{#1, 1}}
\newcommandx{\cteinf}[1][1=\alpha]{L_{#1, 2}}
\newcommandx{\ctesup}[1][1=\alpha]{L_{#1, 3}}
\newcommand{\data}{\mathscr{D}}
\newcommand{\dimlabel}{\mathsf{m}}
\newcommand{\dimU}{\mathsf{q}}
\newcommand{\dimX}{\mathsf{d}}
\newcommand{\dimY}{\mathsf{p}}
\newcommand{\dlim}[1]{\stackrel{\mathcal{L}_{#1}}{\Rightarrow}}
\newcommand{\dmX}{\gamma}
\newcommandx{\diverg}[1][1=\alpha]{D_{#1}}
\newcommandx{\Domain}[1][1=\alpha]{\Delta_{#1}}
\newcommand{\eproof}{\quad \Box}
\newcommand{\eremark}{</WRAP>}
\newcommand{\eqdef}{:=}
\newcommand{\eqlaw}{\stackrel{\mathcal{L}}{=}}
\newcommand{\eqsp}{\;}
\newcommand{\esssup}{\mathrm{essup}}
\newcommand{\finv}{b_f}
\newcommandx{\falpha}[1][1=\alpha]{f_{#1}}
\newcommandx{\aei}[1][1=\alpha]{$(#1, \Gamma)$-}
\newcommandx{\GammaAlpha}[1][1=\alpha]{ \Gamma}
\renewcommand{\geq}{\geqslant}
\newcommandx{\gmuf}[1][1=\mu]{ {g_{#1}}}
\newcommand{\hmu}{\hat{\mu}}
\newcommand{\Id}{\mathrm{Id}}
\newcommand{\img}{\text{Im}}
\newcommand{\indi}[1]{\mathbf{1}_{#1}}
\newcommand{\indiacc}[1]{\mathbf{1}_{\{#1\}}}
\newcommand{\indin}[1]{\mathbf{1}\{#1\}}
\newcommand{\itemm}{\quad \quad \blacktriangleright \;}
\newcommandx{\iteration}[1][1=\alpha]{\mathcal{I}_{#1}}
\newcommandx{\iterationK}[1][1=\alpha]{\hat{\mathcal{I}}_{#1, M}}
%\newcommand{\ker}{\text{Ker}}
\newcommand{\klbck}[2]{\mathrm{K}\lr{#1||#2}}
\newcommandx{\lbd}[2][1=]{
    \ifthenelse{\equal{#1}{}}
    {{\boldsymbol{\lambda}_{#2}}}
    {\lambda_{#1,#2}}
    }
\newcommand{\law}{\mathcal{L}}
\newcommand{\labelinit}{\pi}
\newcommand{\labelkernel}{Q}
\renewcommand{\leq}{\leqslant}
\newcommand{\likelihood}[1][y]{p(\mathscr{D}, #1)}
\newcommand{\llim}{\lim \limits}
\newcommand{\lliminf}{\liminf \limits}
\newcommand{\llimsup}{\limsup \limits}
\newcommand{\lrav}[1]{\left|#1 \right|}
\newcommand{\lr}[1]{\left(#1 \right)}
\newcommand{\lrb}[1]{\left[#1 \right]}
\newcommand{\lrc}[1]{\left\{#1 \right\}}
\newcommand{\lrcb}[1]{\left\{#1 \right\}}
\newcommand{\ltwo}{\mathrm{L}^2}
\newcommand{\Ltwo}{\mathrm{L}^2}
\newcommand{\marginal}{p(\mathscr{D})}
\newcommand{\mc}[1]{\mathcal{#1}}
\newcommand{\mcf}{\mathcal{F}}
\newcommand{\meas}[1]{\mathrm{M}_{#1}}
\newcommand{\muf}{{\mu^\star}}
\newcommand{\mubar}{{\bar{\mu}}}
\newcommand{\nset}{\mathbb N}
\newcommand{\nstar}{\mathbb{N}^\star}
\newcommand{\PE}{\mathbb E}
\newcommandx{\posterior}[1][1=y]{p(#1|\mathscr{D})}
\newcommand{\PP}{\mathbb P}
\newcommand{\PQ}{\mathbb Q}
\newcommandx{\Psifn}[1][1=\alpha]{\Psi_{#1, L}}
\newcommandx{\Psif}[1][1=\alpha]{\Psi_{#1}}
\newcommandx{\Psifk}[1][1=\alpha]{\Psi_{#1, L}}
\renewcommand{\rho}{\varrho}
\newcommand{\rmd}{\mathrm d}
\newcommand{\rme}{\mathrm e}
\newcommand{\rmi}{\mathrm i}
\newcommand{\Rset}{\mathbb{R}}
\newcommand{\rset}{\mathbb{R}}
\newcommand{\rstar}{\mathbb{R}^\star}
\newcommand{\rti}{\sigma}
\newcommand{\set}[2]{\lrc{#1\eqsp: \eqsp #2}}
\newcommand{\settxt}[2]{\{#1\eqsp: \eqsp #2\}}
\newcommand{\simplex}{\mathcal{S}}
\newcommand{\tgamma}{{\tilde{\Gamma}}}
\newcommandx{\thetat}[2][1=j, 2=t]{\theta_{#1,#2}}
\newcommand{\tmu}{{\tilde{\mu}}}
\newcommand{\tv}[1]{\left\|#1\right\|_{TV}}
\newcommand{\tvnorm}[1]{\left\|#1\right\|_{\mathrm{TV}}}
\newcommand{\tq}{\tilde q}
\newcommand{\Tset}{{\mathsf{T}}}
\newcommand{\Tsigma}{\mathcal{T}}
\newcommand{\unif}{\mathrm{Unif}}
\newcommand{\Xset}{\mathsf T}
\newcommand{\Xsigma}{\mathcal T}
\newcommand{\Yset}{\mathsf Y}
\newcommand{\Ysigma}{\mathcal Y}
\newcommand{\Var}{\mathbb{V}\mathrm{ar}}
\newcommand{\zset}{\mathbb{Z}}
\newcommand{\Zset}{\mathsf{Z}}
\newenvironment{encart}[1][]{\ \newline\noindent $\blacktriangleright$ \hrulefill \textcolor{violet}{\textbf{\textsc{ #1: start}}} \hrulefill $\blacktriangleleft$\newline}{\ \newline\noindent $\blacktriangleright$ \hrulefill \textcolor{violet}{\textbf{\textsc{ end }}}\hrulefill $\blacktriangleleft$ \newline} 

\newcommand{\iid}{\stackrel{\mathrm{i.i.d}}{\sim}}
\newcommandx{\dens}[3][1=,2=]%
{
\operatorname{p}_{#1}^{#2}(#3)
}
\newcommandx{\aslim}[1]{\ensuremath{\stackrel{#1 \mbox{-} \text{a.s.}}{\longrightarrow}}}
% \newcommand{\Zset}{\mathsf{Z}}
\newcommand{\Zsigma}{\mathcal{Z}}
\newcommand{\borel}{\mathcal{B}}
\newcommand{\weaklim}{\ensuremath{\stackrel{\mathcal{L}_{\PP}}{\Rightarrow}}}
\newcommand{\eqLaw}{\stackrel{\mathcal L}{=}}
\newcommand{\indep}{\rotatebox[origin=c]{90}{$\models$}}
\newcommandx\functionsetspec[1][1=]{
\ifthenelse{\equal{#1}{c}}{\mathrm{C}}%fonctions continues
{\ifthenelse{\equal{#1}{bc}}{\mathrm{C}_b}%fonctions continues born\'{e}es
{\ifthenelse{\equal{#1}{u}}{\mathrm{U}}%fonctions uniform\'{e}ment continues
{\ifthenelse{\equal{#1}{bu}}{\mathrm{U}_b}%fonctions uniform\'{e}ment continues born\'{e}es
{\ifthenelse{\equal{#1}{l}}{\mathrm{Lip}}%fonctions lipschitz
{\ifthenelse{\equal{#1}{bl}}{\mathrm{Lip}_b}%fonctions lipschitz born\'{e}es
{\mathbb{F}_{#1}}%le reste
}}}}}}
\newcommandx\sequence[3][2=,3=]
{\ifthenelse{\equal{#3}{}}{\ensuremath{\{ #1_{#2}\}}}{\ensuremath{\{ #1_{#2}, \eqsp #2 \in #3 \}}}}
% \newcommand{\Yset}{\mathsf{Y}}
% \newcommand{\Ysigma}{\mathcal{Y}}
\newcommandx\dsequence[4][3=,4=]{\ensuremath{\{ (#1_{#3}, #2_{#3}), \eqsp #3 \in #4 \}}}
% \newcommand{\Id}{\mathrm{I}}
% \newcommand{\lrav}[1]{\left|#1 \right|}
% \newcommand{\rme}{\mathrm{e}}
% \newcommand{\rmi}{\mathrm{i}}
\newcommand{\fracc}[2]{\frac{#1}{#2}}
\newcommand{\bs}{\begin{shaded}}
\newcommand{\es}{\end{shaded}}
\newcommand{\bfr}{\begin{framed}}
\newcommand{\efr}{\end{framed}}
\newcommand{\blb}{\begin{leftbar}}
\newcommand{\elb}{\end{leftbar}}
% \newcommand{\ltwo}{\mathrm{L}_2}
% \newcommand{\dlim}[1]{\stackrel{\mathcal{L}_{#1}}{\Rightarrow}}
\newcommand{\plim}[1]{\stackrel{#1-prob}{\longrightarrow}}
\newcommand{\gauss}{\mathcal{N}}
% \newcommand{\eqsp}{}

\global\long\def\1{\mathbf{1}}%
\global\long\def\as{\PP-\mbox{a.s.}}%
\global\long\def\eqdef{:=}%
\global\long\def\eqlaw{\stackrel{\mathcal{L}}{=}}%
\global\long\def\funcset#1{\mathsf{F_{#1}}}%
\global\long\def\indi#1{\1_{#1}}%
\global\long\def\indiacc#1{\indi{\left\{  #1\right\}  }}%
\global\long\def\lfuncset#1{\mathsf{L_{#1}}}%
\global\long\def\lr#1{\left(#1\right)}%
\global\long\def\PE{\mathbb{E}}%
\global\long\def\lrb#1{\left[#1\right]}%
\global\long\def\lrcb#1{\left\{  #1\right\}  }%
\global\long\def\mcbb{\mathcal{B}}%
\global\long\def\meas#1{\mathsf{M}_{#1}}%
\global\long\def\mh#1{P_{\langle#1\rangle}^{MH}}%
\global\long\def\nset{\mathbb{N}}%
\global\long\def\mc#1{\mathcal{#1}}%
\global\long\def\mcf{\mathcal{F}}%
\global\long\def\mcg{\mathcal{G}}%
\global\long\def\PP{\mathbb{P}}%
\global\long\def\rmd{\mathrm{d}}%
\global\long\def\rset{\mathbb{R}}%
\global\long\def\seq#1#2{\lrcb{#1\,:\,#2}}%
\global\long\def\set#1#2{\lrcb{#1\,:\,#2}}%
\global\long\def\Xset{\mathsf{X}}%
\global\long\def\Xsigma{\mathcal{X}}%





\usetikzlibrary{tikzmark,positioning}

\setbeamercovered{transparent}
\DeclareOptionBeamer{compress}{\beamer@compresstrue}
\ProcessOptionsBeamer
\mode<presentation>

\useoutertheme[footline=authortitle]{miniframes}
\useinnertheme{circles}
\usecolortheme{whale}
\usecolortheme{orchid}
\usetheme{Berkeley}

\definecolor{beamer@blendedblue}{rgb}{0.137,0.466,0.741}

\setbeamercolor{structure}{fg=beamer@blendedblue}
\setbeamercolor{titlelike}{parent=structure}
\setbeamercolor{frametitle}{fg=black}
\setbeamercolor{title}{fg=black}
\setbeamercolor{item}{fg=black}

\mode
<all>

\usefonttheme{professionalfonts}

\setbeamertemplate{footline}{%
  \begin{beamercolorbox}[colsep=1.5pt]{upper separation line foot}
  \end{beamercolorbox}
  \begin{beamercolorbox}[ht=2.5ex,dp=1.125ex,%
    leftskip=.3cm,rightskip=.3cm plus1fil]{title in head/foot}%
    \leavevmode{
    {\usebeamerfont{author in head/foot}\usebeamercolor[fg]{author in head/foot}\insertshortauthor  \,,\ Telecom Sudparis} \quad \quad \quad \quad \quad \quad \quad \quad \quad
    \usebeamerfont{title in head/foot}\insertshorttitle}%
    \hfill%
    {\usebeamerfont{author in head/foot}\usebeamercolor[fg]{author in head/foot}\insertframenumber~/~\inserttotalframenumber}% NEW
  \end{beamercolorbox}%
  \begin{beamercolorbox}[colsep=1.5pt]{lower separation line foot}
  \end{beamercolorbox}
}

\setbeamertemplate{headline}{}
\setbeamertemplate{navigation symbols}{}

\makeatletter
\newcommand{\sub}[1][]{%
    \beamer@ifempty{#1}{}{\def\beamer@defaultospec{#1}}%
    \ifnum \@itemdepth >2\relax\@toodeep\else
    \advance\@itemdepth\@ne
    \beamer@computepref\@itemdepth% sets \beameritemnestingprefix
    \usebeamerfont{itemize/enumerate \beameritemnestingprefix body}%
    \usebeamercolor[fg]{itemize/enumerate \beameritemnestingprefix body}%
    \usebeamertemplate{itemize/enumerate \beameritemnestingprefix body begin}%
    \setlength{\csname leftmargin\romannumeral\@itemdepth\endcsname}{.5em}%
    \list{}%
     {\def\makelabel##1{%
      {%
      \hss\llap{{%
       \usebeamerfont*{itemize \beameritemnestingprefix item}%
       \usebeamercolor[fg]{itemize \beameritemnestingprefix item}##1}}%
      }%
     }%
     }
    \fi%
    \beamer@cramped%
    \raggedright%
    \beamer@firstlineitemizeunskip%
}
\def\endsub{\enditemize}
\makeatother

\newcommand\Wider[2][3em]{%
\makebox[\linewidth][c]{%
  \begin{minipage}{\dimexpr\textwidth+#1\relax}
  \raggedright#2
  \end{minipage}%
  }%
}
\newcommand\mycol[1]{{\color{red}#1}}
\newcommand\mycoltwo[1]{{\color{blue}#1}}
\newcommand\mycolthree[1]{{\color{Emerald}#1}}
\newcommand\mycolm[1]{{\color{red}#1}}
\newcommand\colbox[1]{\colorbox{Yellow}{#1}}
\newcommand{\coloredeq}[1]{\begin{empheq}[box=\colorbox{yellow}]{align}#1\end{empheq}}
\newcommand{\coloredeqstar}[1]{\begin{empheq}[box=\colorbox{yellow}]{align*}#1\end{empheq}}

\title[MCMC: Theory and Applications]{Markov Chain Monte Carlo\\ \textsl{Theory and Practical applications}\\ \colbox{Chapter 4}: \alert{Geometric ergodicity and CLT}}
\author{R. Douc and S. Le Corff}

\institute{T\'el\'ecom SudParis, Institut Polytechnique de Paris \\ randal.douc@telecom-sudparis.eu \vspace{0.1cm}}

\date[\today]{
\includegraphics[scale=0.1]{{logoIMT}.png}
}

\begin{document}
\frame{\titlepage}


\begin{frame}{Outline}
    \tableofcontents[sectionstyle=show, subsectionstyle=show/show/hide]
\end{frame}

\AtBeginSection[]{%
\begin{frame}{Outline}
    \tableofcontents[currentsection, subsectionstyle=show/show/hide]
\end{frame}
}






\section{Introduction}

\begin{frame}
    \colbox{Geometric ergodicity} means that there exists constants $C>0$ and $\rho\in(0,1)$ such that for all $n\in\nset$, 
    $$
    \mycoltwo{\tvnorm{\mu P^n -\pi} \leq C \rho^n}
    $$
    where $\tvnorm{\cdot}$ is the \alert{total variation norm} (to be defined later) between two measures. 
    \pause
    \begin{enumerate}
        \item $\mu P^n$ is the law of $X_n$ starting from $X_0\sim \mu$ \pause
        \item $\pi$ is the law of $X_n$ starting from $X_0\sim \pi$\pause
        \item \alert{Geometric ergodicity for Markov chains} should not be confused with the notion of \mycoltwo{ergodic dynamical systems}
    \end{enumerate}
    \pause
    \colbox{CLT} means that  
    $$
    \mycoltwo{n^{-1/2} \sum_{k=0}^{n-1} \{h(X_k)-\pi(h)\} \dlim{\PP_\pi} \gauss(0,\sigma_\pi^2(h))}
    $$
    where $h$ belongs to some class of functions and $\sigma_\pi$ should be explicit. 
\end{frame}


\section{Coupling and total variation}
\begin{frame}
    \frametitle{What is a coupling?}
    \begin{definition}
        Let $(\Xset,\Xsigma)$ be a measurable space and let $\nu,\mu$ be two probability measures $\mu,\nu \in \meas 1(\Xset)$. We define $\coupling(\mu,\nu)$, the \colbox{coupling set} associated to $(\mu,\nu)$ as follows
    $$ 
\alert{    \coupling(\mu,\nu)=\set{\gamma\in\meas 1(\Xset^2)}{\gamma(\cdot\times\Xset)=\mu(\cdot), \gamma(\Xset\times \cdot)=\nu(\cdot)}}     
    $$
       Any $\gamma\in\coupling(\mu,\nu)$ is called a \colbox{coupling} of $(\mu,\nu)$.
        \end{definition}
        \pause
        \begin{enumerate}
            \item In words, $\gamma$ is a \colbox{coupling of $(\mu,\nu)$} if the following property holds: if $(X,Y) \sim \gamma$, then we have the \colbox{marginal conditions}: $X\sim\mu$ and $Y\sim \nu$. \pause 
            \item \colbox{Example:} The law of \alert{$(X,X)$} where $X\sim \mu$ is a coupling of $(\mu,\mu)$. Other example if $X\sim \mu$ and $Y\sim \mu$ and $X \perp Y$, then, the law of \alert{$(X,Y)$} is a coupling of $(\mu,\mu)$. 
        \end{enumerate}
\end{frame}


\begin{frame}
    \frametitle{What are the different expressions of the total variation norm?}
    \begin{definition} \label{def:totvar}
        \index{total variation distance}
        Let $(\Xset,\Xsigma)$ be a measurable space and let $\nu,\mu$ be two probability measures $\mu,\nu \in \meas 1(\Xset)$. Then the \colbox{total variation norm} between $\mu$ and $\nu$ noted $\tvnorm{\mu-\nu}$, is defined by
        \begin{align}
        \tvnorm{\mu-\nu}&=\alert{2\sup\set{|\mu(f)-\nu(f)|}{f\in \funcset{}(\Xset), 0\leq f\leq 1}} \label{eq:def:totvar:one}\\
                &=\mycoltwo{\int |\varphi_0-\varphi_1|(x) \zeta(\rmd x)} \label{eq:def:totvar:two}\\
                &=\textcolor{brown}{2\inf\set{\PP(X\neq Y) }{ (X,Y) \sim \gamma \mbox{ where } \gamma \in \coupling(\mu,\nu)}} \label{eq:def:totvar:three}
        \end{align}
        where \colbox{$\mu(\rmd x)=\varphi_0(x) \zeta(\rmd x)$} and \colbox{$\nu(\rmd x)=\varphi_1(x) \zeta(\rmd x)$}.
        \end{definition}
        \mycolthree{Proof is given in the lecture notes}
\end{frame}

\section{Geometric ergodicity}

\begin{frame}
    \frametitle{Two types of assumptions}

        \begin{block}{Assumption A1}
         [\colbox{\bf Minorizing condition}] for all $d>0$, there exists $\epsilon_d>0$ and a probability measure $\nu_d$ such that \index{minorizing condition}
        \begin{equation}
        \label{eq:minor}
        \alert{\forall x\in C_d\eqdef\{V\leq d\}, \quad
        P(x,\cdot) \geq \epsilon_d \nu_d(\cdot)}
        \end{equation}
        \end{block}
        \pause
        \begin{block}{Assumption A2}
         [\colbox{\bf Drift condition}] \index{drift condition}there exists a constants $(\lambda,b)\in (0,1) \times \rset^+$ such that for all $x\in\Xset$,
        $$
        \alert{PV(x) \leq \lambda V(x)+b}
        $$
        \end{block}
        

\end{frame}
\begin{frame}
    \frametitle{Geometric ergodicity and initial condition}
    \begin{theorem}{\colbox{\textbf{(Forgetting of the initialization)}}}\label{thm:ergo:geom}  \index{geometric ergodicity}
        Assume (A1) and (A2)\  for some measurable function $V\geq1$.
        \pause
        Then, there exists a constant $\varrho \in (0,1)$ such that for all $x,x'\in\Xset$ and all $n\in\nset$,
        $$
        \alert{\tvnorm{P^n(x,\cdot)-P^n(x',\cdot)} \leq \varrho^n \lrb{V(x)+V(x')}}  .
        $$
        \end{theorem}
       \mycolthree{Proof is hard. It is given in the lecture notes.} 
\end{frame}
\begin{frame}
    \frametitle{The geometric ergodicity theorem}
    \begin{corollary}{\colbox{\bf (Geometric ergodicity)}} \label{cor:ergod}
        Assume that (A1) and (A2)\ hold for some measurable function $V\geq 1$. \pause Then, the Markov kernel $P$ admits a \mycoltwo{unique invariant probability measure $\pi$}. \pause Moreover, $\pi(V)<\infty$ and there exists constants $(\varrho,\alpha) \in (0,1) \times \rset^+$ such that for all $\mu\in\meas 1(\Xset)$ and all $n\in\nset$,
        $$
        \alert{\tvnorm{\mu P^n-\pi} \leq \alpha \varrho^n \mu(V)}  .
        $$
        \end{corollary}
        \mycolthree{Proof should be done on the blackboard}
    

\end{frame}
\section{Central Limit theorem}
\subsection{Recap on martingales}

\begin{frame}
    % \frametitle{Refresh on Martingales}
    Let $(M_n)_{n \in \nset}$ be a sequence of random variables on the same probability space $(\Omega,\mcf,\PP)$ and let $(\mcf_n)_{n\in\nset}$ be a filtration (ie for all $n\in\nset$, $\mcf_n\subset \mcf_{n+1}\subset \mcf$). 
    \begin{definition}
        We say that $(M_n)_{n \in \nset}$ is a \colbox{\rm $(\mcf_n)$-martingale} if for all $n\in\nset$, $M_n$ is integrable and for all $n\geq 1$,
    $$
    \alert{\PE[M_n|\mcf_{n-1}]=M_{n-1}}
    $$
    \pause 
    The \colbox{\em increment process} of the martingale is by definition $(M_{n+1}-M_n)_{n\in\nset}$.
    \end{definition}
    \pause    
    \begin{theorem} \label{thm:clt:marting} \index{martingales!central limit theorem}
    If a sequence $(M_n)_{n \in \nset}$ is a \colbox{$(\mcf_n)$-martingale with stationary} \colbox{and square integrable increments}, then
    $$
    \alert{n^{-1/2} M_n \dlim{\PP} \gauss\lr{0,\PE[(M_1-M_0)^2]}}
    $$
    \end{theorem}
\end{frame}

\subsection{The Poisson equation}
\begin{frame}
    \frametitle{What is the Poisson equation ?}
    \begin{definition} \index{Poisson equation}
        For a given measurable function $h$ such that $\pi|h|<\infty$, the \colbox{Poisson equation} is defined by
        \begin{equation} \label{eq:Poisson}
        \alert{\hat h - P \hat h=h-\pi(h)}
        \end{equation}
        A solution to the Poisson equation is a function $\hat h$ for which \eqref{eq:Poisson} holds provided that $P|\hat h|(x)<\infty$ for all $x\in\Xset$.
        
        \end{definition}
\end{frame}
\begin{frame}
    \frametitle{Link between Poisson equations and Martingales}
    \begin{center}
        \mycoltwo{ \bf Link between Poisson equations and Martingales}
    \end{center}Define
    \begin{align*}
        S_n(h)&=\sum_{k=0}^{n-1} \lrcb{h(X_k)-\pi(h)} \pause\\
        &=M_n(\hat h)+\hat h(X_0)-\hat h(X_n)    
    \end{align*}
    \pause
    where
    \begin{equation}\label{eq:def:marting}
    \alert{M_n(\hat h)=\sum_{k=1}^n \lrcb{\hat h(X_k)-P\hat h(X_{k-1})}}
    \end{equation}
    \pause
    Note that $\lrcb{M_n(\hat h)}_{n\in\nset}$ is indeed a \colbox{$(\mcf_k)$-martingale} where $\mcf_k=\sigma(X_0,\ldots,X_k)$. \pause Indeed we have
    \mycoltwo{\begin{align*}
        \PE[M_n(\hat h)|\mcf_{n-1}]-M_{n-1}(h)&=\PE[\hat h(X_n)-P\hat h(X_{n-1})|\mcf_{n-1}]\\
        &=P\hat h(X_{n-1})-P\hat h(X_{n-1})=0
    \end{align*}}
    
    

\end{frame}

\begin{frame}
    \frametitle{How can we express the solution of a Poisson equation?}
    \begin{theorem} \label{thm:solPoiss}
        Assume that (A1) and (A2) hold for some measurable function $V\geq 1$. Then, for any function $h$ such that $|h|\leq V$, the function
       \begin{equation}\label{eq:solPoisson}
        \alert{\hat h=\sum_{n=0}^\infty \lrcb{P^n h-\pi(h)}}
       \end{equation}
       is well-defined. Moreover, $\hat h$ is a \colbox{solution of the Poisson} \colbox{ equation} associated to $h$ and there exists a constant $\gamma$ such that  for all $x\in\Xset$,
       $$
       \alert{|\hat h(x)|\leq \gamma V(x)}
       $$
        \end{theorem}
        \mycolthree{Proof should be done on the blackboard}
\end{frame}

\begin{frame}
    \frametitle{CLT}
    \begin{theorem}{\colbox{\bf (CLT with Poisson assumption)}}
        \label{thm:clt-poisson} \index{central limit theorem}
        Let $P$ be a  Markov kernel with a unique invariant probability measure $\pi$. Let
        $h \in \ltwo(\pi)$. Assume that there exists a solution
        \colbox{$\hat{h} \in \ltwo(\pi)$} to the Poisson equation $\hat{h} - P \hat{h} = h$. \pause Then
        \begin{align*}
            \alert{n^{-1/2} \sum_{k=0}^{n-1} \{h(X_k)-\pi(h)\} \dlim{\PP_\pi} \gauss(0,\sigma_\pi^2(h))} \eqsp ,
        \end{align*}
        where
        \begin{equation}
        \label{eq:equality-variance}
        \alert{\sigma_\pi^2(h) = \PE_\pi[ \{\hat{h}(X_1)-P\hat{h}(X_0)\}^2]} %= \pi(\hat{h}^2 - (P\hat{h})^2)
      %  = 2 \pi(h\hat{h})-\pi(h^2) \eqsp.
        \end{equation}
      \end{theorem}
      \mycolthree{Proof should be done on the blackboard}
    

\end{frame}
\subsection{Central limit theorems}
\begin{frame}
    \frametitle{CLT}
    \begin{theorem}{\colbox{\bf (CLT with A1-A2 assumptions)}} \label{thm:clrHairer}
        Assume that (A1 and (A2)\ hold for some function $V$. \pause Then, for all measurable functions $h$ such that $|h|^2\leq V$,
          \begin{align*}
            \alert{n^{-1/2} \sum_{k=0}^{n-1} \{h(X_k)-\pi(h)\}  \dlim{\PP_\pi} \gauss(0,\sigma_\pi^2(h))} \eqsp ,
          \end{align*}
          where
          \begin{equation}
          \label{eq:equality-variance}
          \alert{\sigma_\pi^2(h) = \PE_\pi[ \{\hat{h}(X_1)-P\hat{h}(X_0)\}^2]} %= \pi(\hat{h}^2 - (P\hat{h})^2)
        %  = 2 \pi(h\hat{h})-\pi(h^2) \eqsp.
          \end{equation}
        and $\hat h$ is defined as in \eqref{eq:solPoisson}.
        \end{theorem}
        
        \mycolthree{Proof should be done on the blackboard} 

\end{frame}
\end{document} 