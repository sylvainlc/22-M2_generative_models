\documentclass[usenames,dvipsnames]{beamer}

\RequirePackage[OT1]{fontenc}
\RequirePackage{amsthm,amsmath}
\RequirePackage[numbers]{natbib}
\resetcounteronoverlays{algocf}
\usepackage{appendixnumberbeamer}

\usepackage[utf8]{inputenc}
\usepackage[english]{babel}
\usepackage[ruled]{algorithm2e}
\usepackage{amsmath}
\usepackage{amsfonts}
\usepackage{amssymb}
\usepackage{changepage}
\usepackage{empheq}
\usepackage{mathrsfs}
\usepackage{tikz}
\usepackage{transparent}
\usepackage{xcolor}
\usepackage{xargs}

\usepackage{pifont}% http://ctan.org/pkg/pifont
\newcommand{\cmark}{\ding{51}}%
\newcommand{\xmark}{\ding{55}}%

\newcommand{\kd}[1]{\todo[color=red!25,size=\footnotesize]{ \textbf{KD:}  #1}}
\newcommand{\rd}[1]{\todo[color=blue!40,size=\footnotesize]{ \textbf{RD:}  #1}}
\newcommand{\fr}[1]{\todo[color=orange!40,size=\footnotesize]{ \textbf{FrR:}  #1}}
\newcommand{\fp}[1]{\todo[color=yellow!40,size=\footnotesize]{ \textbf{FrP:}  #1}}
\newcommandx{\admiss}[1][1=f]{\mathsf{A}_{#1}}
\newcommand{\arginf}{\mathrm{arginf}}
\newcommand{\argmin}{\mathrm{argmin}}
\newcommand{\argmax}{\mathrm{argmax}}
\newcommand{\argsup}{\mathrm{argsup}}
\newcommand{\as}{\mathrm{a.s.}}
%\newcommand{\b}[1]{{\mathbf{#1}}}
\newcommand{\ball}{\mathsf{B}}
\newcommandx{\binfty}[1][1=\alpha]{|b|_{\infty, #1}}
\newcommandx{\bmuf}[2][1=\mu, 2=\alpha]{ {b_{#1, #2}}}
\newcommandx{\bmufk}[2][1=\mu, 2=\alpha]{\hat{b}_{#1, #2, M}}
\newcommand{\bproof}{\textbf{Proof :}\quad}
\newcommand{\card}{\mathrm{card}}
\newcommand{\chunk}[3]{{#1}_{#2:#3}}
\newcommandx{\couple}[2][1=\PQ, 2=\PP]{(#1 || #2)}
\newcommand{\coupling}{\mathcal{C}}
\newcommand{\Cov}{\mathbb{C}\mathrm{ov}}
\newcommand{\cte}{\kappa}
\newcommandx{\ctemono}[1][1=\alpha]{L_{#1, 1}}
\newcommandx{\cteinf}[1][1=\alpha]{L_{#1, 2}}
\newcommandx{\ctesup}[1][1=\alpha]{L_{#1, 3}}
\newcommand{\data}{\mathscr{D}}
\newcommand{\dimlabel}{\mathsf{m}}
\newcommand{\dimU}{\mathsf{q}}
\newcommand{\dimX}{\mathsf{d}}
\newcommand{\dimY}{\mathsf{p}}
\newcommand{\dlim}[1]{\stackrel{\mathcal{L}_{#1}}{\Rightarrow}}
\newcommand{\dmX}{\gamma}
\newcommandx{\diverg}[1][1=\alpha]{D_{#1}}
\newcommandx{\Domain}[1][1=\alpha]{\Delta_{#1}}
\newcommand{\eproof}{\quad \Box}
\newcommand{\eremark}{</WRAP>}
\newcommand{\eqdef}{:=}
\newcommand{\eqlaw}{\stackrel{\mathcal{L}}{=}}
\newcommand{\eqsp}{\;}
\newcommand{\esssup}{\mathrm{essup}}
\newcommand{\finv}{b_f}
\newcommandx{\falpha}[1][1=\alpha]{f_{#1}}
\newcommandx{\aei}[1][1=\alpha]{$(#1, \Gamma)$-}
\newcommandx{\GammaAlpha}[1][1=\alpha]{ \Gamma}
\renewcommand{\geq}{\geqslant}
\newcommandx{\gmuf}[1][1=\mu]{ {g_{#1}}}
\newcommand{\hmu}{\hat{\mu}}
\newcommand{\Id}{\mathrm{Id}}
\newcommand{\img}{\text{Im}}
\newcommand{\indi}[1]{\mathbf{1}_{#1}}
\newcommand{\indiacc}[1]{\mathbf{1}_{\{#1\}}}
\newcommand{\indin}[1]{\mathbf{1}\{#1\}}
\newcommand{\itemm}{\quad \quad \blacktriangleright \;}
\newcommandx{\iteration}[1][1=\alpha]{\mathcal{I}_{#1}}
\newcommandx{\iterationK}[1][1=\alpha]{\hat{\mathcal{I}}_{#1, M}}
%\newcommand{\ker}{\text{Ker}}
\newcommand{\klbck}[2]{\mathrm{K}\lr{#1||#2}}
\newcommandx{\lbd}[2][1=]{
    \ifthenelse{\equal{#1}{}}
    {{\boldsymbol{\lambda}_{#2}}}
    {\lambda_{#1,#2}}
    }
\newcommand{\law}{\mathcal{L}}
\newcommand{\labelinit}{\pi}
\newcommand{\labelkernel}{Q}
\renewcommand{\leq}{\leqslant}
\newcommand{\likelihood}[1][y]{p(\mathscr{D}, #1)}
\newcommand{\llim}{\lim \limits}
\newcommand{\lliminf}{\liminf \limits}
\newcommand{\llimsup}{\limsup \limits}
\newcommand{\lrav}[1]{\left|#1 \right|}
\newcommand{\lr}[1]{\left(#1 \right)}
\newcommand{\lrb}[1]{\left[#1 \right]}
\newcommand{\lrc}[1]{\left\{#1 \right\}}
\newcommand{\lrcb}[1]{\left\{#1 \right\}}
\newcommand{\ltwo}{\mathrm{L}^2}
\newcommand{\Ltwo}{\mathrm{L}^2}
\newcommand{\marginal}{p(\mathscr{D})}
\newcommand{\mc}[1]{\mathcal{#1}}
\newcommand{\mcf}{\mathcal{F}}
\newcommand{\meas}[1]{\mathrm{M}_{#1}}
\newcommand{\muf}{{\mu^\star}}
\newcommand{\mubar}{{\bar{\mu}}}
\newcommand{\nset}{\mathbb N}
\newcommand{\nstar}{\mathbb{N}^\star}
\newcommand{\PE}{\mathbb E}
\newcommandx{\posterior}[1][1=y]{p(#1|\mathscr{D})}
\newcommand{\PP}{\mathbb P}
\newcommand{\PQ}{\mathbb Q}
\newcommandx{\Psifn}[1][1=\alpha]{\Psi_{#1, L}}
\newcommandx{\Psif}[1][1=\alpha]{\Psi_{#1}}
\newcommandx{\Psifk}[1][1=\alpha]{\Psi_{#1, L}}
\renewcommand{\rho}{\varrho}
\newcommand{\rmd}{\mathrm d}
\newcommand{\rme}{\mathrm e}
\newcommand{\rmi}{\mathrm i}
\newcommand{\Rset}{\mathbb{R}}
\newcommand{\rset}{\mathbb{R}}
\newcommand{\rstar}{\mathbb{R}^\star}
\newcommand{\rti}{\sigma}
\newcommand{\set}[2]{\lrc{#1\eqsp: \eqsp #2}}
\newcommand{\settxt}[2]{\{#1\eqsp: \eqsp #2\}}
\newcommand{\simplex}{\mathcal{S}}
\newcommand{\tgamma}{{\tilde{\Gamma}}}
\newcommandx{\thetat}[2][1=j, 2=t]{\theta_{#1,#2}}
\newcommand{\tmu}{{\tilde{\mu}}}
\newcommand{\tv}[1]{\left\|#1\right\|_{TV}}
\newcommand{\tvnorm}[1]{\left\|#1\right\|_{\mathrm{TV}}}
\newcommand{\tq}{\tilde q}
\newcommand{\Tset}{{\mathsf{T}}}
\newcommand{\Tsigma}{\mathcal{T}}
\newcommand{\unif}{\mathrm{Unif}}
\newcommand{\Xset}{\mathsf T}
\newcommand{\Xsigma}{\mathcal T}
\newcommand{\Yset}{\mathsf Y}
\newcommand{\Ysigma}{\mathcal Y}
\newcommand{\Var}{\mathbb{V}\mathrm{ar}}
\newcommand{\zset}{\mathbb{Z}}
\newcommand{\Zset}{\mathsf{Z}}
\newenvironment{encart}[1][]{\ \newline\noindent $\blacktriangleright$ \hrulefill \textcolor{violet}{\textbf{\textsc{ #1: start}}} \hrulefill $\blacktriangleleft$\newline}{\ \newline\noindent $\blacktriangleright$ \hrulefill \textcolor{violet}{\textbf{\textsc{ end }}}\hrulefill $\blacktriangleleft$ \newline} 

\newcommand{\iid}{\stackrel{\mathrm{i.i.d}}{\sim}}
\newcommandx{\dens}[3][1=,2=]%
{
\operatorname{p}_{#1}^{#2}(#3)
}
\newcommandx{\aslim}[1]{\ensuremath{\stackrel{#1 \mbox{-} \text{a.s.}}{\longrightarrow}}}
% \newcommand{\Zset}{\mathsf{Z}}
\newcommand{\Zsigma}{\mathcal{Z}}
\newcommand{\borel}{\mathcal{B}}
\newcommand{\weaklim}{\ensuremath{\stackrel{\mathcal{L}_{\PP}}{\Rightarrow}}}
\newcommand{\eqLaw}{\stackrel{\mathcal L}{=}}
\newcommand{\indep}{\rotatebox[origin=c]{90}{$\models$}}
\newcommandx\functionsetspec[1][1=]{
\ifthenelse{\equal{#1}{c}}{\mathrm{C}}%fonctions continues
{\ifthenelse{\equal{#1}{bc}}{\mathrm{C}_b}%fonctions continues born\'{e}es
{\ifthenelse{\equal{#1}{u}}{\mathrm{U}}%fonctions uniform\'{e}ment continues
{\ifthenelse{\equal{#1}{bu}}{\mathrm{U}_b}%fonctions uniform\'{e}ment continues born\'{e}es
{\ifthenelse{\equal{#1}{l}}{\mathrm{Lip}}%fonctions lipschitz
{\ifthenelse{\equal{#1}{bl}}{\mathrm{Lip}_b}%fonctions lipschitz born\'{e}es
{\mathbb{F}_{#1}}%le reste
}}}}}}
\newcommandx\sequence[3][2=,3=]
{\ifthenelse{\equal{#3}{}}{\ensuremath{\{ #1_{#2}\}}}{\ensuremath{\{ #1_{#2}, \eqsp #2 \in #3 \}}}}
% \newcommand{\Yset}{\mathsf{Y}}
% \newcommand{\Ysigma}{\mathcal{Y}}
\newcommandx\dsequence[4][3=,4=]{\ensuremath{\{ (#1_{#3}, #2_{#3}), \eqsp #3 \in #4 \}}}
% \newcommand{\Id}{\mathrm{I}}
% \newcommand{\lrav}[1]{\left|#1 \right|}
% \newcommand{\rme}{\mathrm{e}}
% \newcommand{\rmi}{\mathrm{i}}
\newcommand{\fracc}[2]{\frac{#1}{#2}}
\newcommand{\bs}{\begin{shaded}}
\newcommand{\es}{\end{shaded}}
\newcommand{\bfr}{\begin{framed}}
\newcommand{\efr}{\end{framed}}
\newcommand{\blb}{\begin{leftbar}}
\newcommand{\elb}{\end{leftbar}}
% \newcommand{\ltwo}{\mathrm{L}_2}
% \newcommand{\dlim}[1]{\stackrel{\mathcal{L}_{#1}}{\Rightarrow}}
\newcommand{\plim}[1]{\stackrel{#1-prob}{\longrightarrow}}
\newcommand{\gauss}{\mathcal{N}}
% \newcommand{\eqsp}{}

\global\long\def\1{\mathbf{1}}%
\global\long\def\as{\PP-\mbox{a.s.}}%
\global\long\def\eqdef{:=}%
\global\long\def\eqlaw{\stackrel{\mathcal{L}}{=}}%
\global\long\def\funcset#1{\mathsf{F_{#1}}}%
\global\long\def\indi#1{\1_{#1}}%
\global\long\def\indiacc#1{\indi{\left\{  #1\right\}  }}%
\global\long\def\lfuncset#1{\mathsf{L_{#1}}}%
\global\long\def\lr#1{\left(#1\right)}%
\global\long\def\PE{\mathbb{E}}%
\global\long\def\lrb#1{\left[#1\right]}%
\global\long\def\lrcb#1{\left\{  #1\right\}  }%
\global\long\def\mcbb{\mathcal{B}}%
\global\long\def\meas#1{\mathsf{M}_{#1}}%
\global\long\def\mh#1{P_{\langle#1\rangle}^{MH}}%
\global\long\def\nset{\mathbb{N}}%
\global\long\def\mc#1{\mathcal{#1}}%
\global\long\def\mcf{\mathcal{F}}%
\global\long\def\mcg{\mathcal{G}}%
\global\long\def\PP{\mathbb{P}}%
\global\long\def\rmd{\mathrm{d}}%
\global\long\def\rset{\mathbb{R}}%
\global\long\def\seq#1#2{\lrcb{#1\,:\,#2}}%
\global\long\def\set#1#2{\lrcb{#1\,:\,#2}}%
\global\long\def\Xset{\mathsf{X}}%
\global\long\def\Xsigma{\mathcal{X}}%





\usetikzlibrary{tikzmark,positioning}

\setbeamercovered{transparent}
\DeclareOptionBeamer{compress}{\beamer@compresstrue}
\ProcessOptionsBeamer
\mode<presentation>

\useoutertheme[footline=authortitle]{miniframes}
\useinnertheme{circles}
\usecolortheme{whale}
\usecolortheme{orchid}

\definecolor{beamer@blendedblue}{rgb}{0.137,0.466,0.741}

\setbeamercolor{structure}{fg=beamer@blendedblue}
\setbeamercolor{titlelike}{parent=structure}
\setbeamercolor{frametitle}{fg=black}
\setbeamercolor{title}{fg=black}
\setbeamercolor{item}{fg=black}

\mode
<all>

\usefonttheme{professionalfonts}

\setbeamertemplate{footline}{%
  \begin{beamercolorbox}[colsep=1.5pt]{upper separation line foot}
  \end{beamercolorbox}
  \begin{beamercolorbox}[ht=2.5ex,dp=1.125ex,%
    leftskip=.3cm,rightskip=.3cm plus1fil]{title in head/foot}%
    \leavevmode{
    {\usebeamerfont{author in head/foot}\usebeamercolor[fg]{author in head/foot}\insertshortauthor  \,,\ Telecom Sudparis} \quad \quad \quad \quad \quad \quad \quad \quad \quad
    \usebeamerfont{title in head/foot}\insertshorttitle}%
    \hfill%
    {\usebeamerfont{author in head/foot}\usebeamercolor[fg]{author in head/foot}\insertframenumber~/~\inserttotalframenumber}% NEW
  \end{beamercolorbox}%
  \begin{beamercolorbox}[colsep=1.5pt]{lower separation line foot}
  \end{beamercolorbox}
}

\setbeamertemplate{headline}{}
\setbeamertemplate{navigation symbols}{}

\makeatletter
\newcommand{\sub}[1][]{%
    \beamer@ifempty{#1}{}{\def\beamer@defaultospec{#1}}%
    \ifnum \@itemdepth >2\relax\@toodeep\else
    \advance\@itemdepth\@ne
    \beamer@computepref\@itemdepth% sets \beameritemnestingprefix
    \usebeamerfont{itemize/enumerate \beameritemnestingprefix body}%
    \usebeamercolor[fg]{itemize/enumerate \beameritemnestingprefix body}%
    \usebeamertemplate{itemize/enumerate \beameritemnestingprefix body begin}%
    \setlength{\csname leftmargin\romannumeral\@itemdepth\endcsname}{.5em}%
    \list{}%
     {\def\makelabel##1{%
      {%
      \hss\llap{{%
       \usebeamerfont*{itemize \beameritemnestingprefix item}%
       \usebeamercolor[fg]{itemize \beameritemnestingprefix item}##1}}%
      }%
     }%
     }
    \fi%
    \beamer@cramped%
    \raggedright%
    \beamer@firstlineitemizeunskip%
}
\def\endsub{\enditemize}
\makeatother

\newcommand\Wider[2][3em]{%
\makebox[\linewidth][c]{%
  \begin{minipage}{\dimexpr\textwidth+#1\relax}
  \raggedright#2
  \end{minipage}%
  }%
}
\newcommand\mycol[1]{{\color{red}#1}}
\newcommand\mycoltwo[1]{{\color{blue}#1}}
\newcommand\mycolthree[1]{{\color{Emerald}#1}}
\newcommand\mycolm[1]{{\color{red}#1}}
\newcommand\colbox[1]{\colorbox{Yellow}{#1}}
\newcommand{\coloredeq}[1]{\begin{empheq}[box=\colorbox{yellow}]{align}#1\end{empheq}}
\newcommand{\coloredeqstar}[1]{\begin{empheq}[box=\colorbox{yellow}]{align*}#1\end{empheq}}

\title[M2DS: MCMC Theory and Applications]{Markov Chain Monte Carlo\\ \textsl{Theory and Practical applications}\\ \colbox{Chapters 2 and 3}}
\author{Randal Douc and Sylvain Le Corff}

\institute{T\'el\'ecom SudParis, Institut Polytechnique de Paris \\ randal.douc@telecom-sudparis.eu \vspace{0.1cm}}

\date[\today]{
\includegraphics[scale=0.1]{{logoIMT}.png}
}

\begin{document}
\frame{\titlepage}


\begin{frame}{Outline}
    \tableofcontents[sectionstyle=show, subsectionstyle=show/show/hide]
\end{frame}

\AtBeginSection[]{%
\begin{frame}{Outline}
    \tableofcontents[currentsection, subsectionstyle=show/show/hide]
\end{frame}
}






\section{Chap 2: Some recaps}

\begin{frame}
Recall that a Markov kernel $P$ is 
\begin{enumerate}
  \item $\pi$-invariant if $\pi P=\pi$ \pause
  \item $\pi$-reversible if $\pi(\rmd x)P(x,\rmd y)=\pi(\rmd y)P(y,\rmd x)$ \pause
  \item $\pi$-reversible implies $\pi$-invariance. 
\end{enumerate}
\pause
\begin{block}{The Metropolis-Hastings algorithm}
\textbf{Input:} $n$\\
\noindent \textbf{Output:} $X_0,\ldots,X_n$\\
\begin{itemize}
  \item At $t=0$, draw $X_{0}$ according to some arbitrary distribution \pause
  \item For $t\leftarrow 0$ to $n-1$
  \begin{enumerate}
    \item Draw independently $Y_{t+1}\sim \colbox{Q}(X_{t},\cdot)$ and $U_{t+1}\sim\mathrm{Unif}(0,1)$ \pause
    \item Set $X_{t+1}=\begin{cases} Y_{t+1} & \mbox{if }U_{t+1}\leq\alpha(X_{t},Y_{t+1})\\ X_{t} & \mbox{otherwise} \end{cases}$
  \end{enumerate}
  where \colbox{$\alpha(x,y)=\alpha^{MH}(x,y)=\min\lr{\frac{\pi(y)q(y,x)}{\pi(x)q(x,y)},1}$}
\end{itemize}  
\end{block}
\end{frame}


\begin{frame}
  The Markov kernel associated to $\seq{X_n}{n\in\nset}$ is given by
$$
  \mycoltwo{\mh{\pi,Q}(x,\rmd y)=Q(x,\rmd y)\alpha(x,y)+\bar{\alpha}(x)\delta_{x}(\rmd y)}.\label{eq:def:mh}
$$  
where $\bar{\alpha}(x)=1-\int_{\Xset} Q(x,\rmd y)\alpha(x,y)$. 
\pause 

\begin{block}{Lemma}
  If the \colbox{detailed balance condition} 
  \begin{equation}
    \label{eq:detailed}
    \alert{\pi(\rmd x) Q(x,\rmd y)\alpha(x,y)=\pi(\rmd y)Q(y,\rmd x) \alpha(y,x)}  
  \end{equation}
  is satisfied, then 
  $\mh{\pi,Q}$ is $\pi$-reversible and hence, $\pi$-invariant.    
\end{block}
\pause 
\begin{itemize}
  \item \fbox{\mycoltwo{$\alpha^{MH}(x,y)=\min\lr{\frac{\pi(y)q(y,x)}{\pi(x)q(x,y)},1}$}} or 
  \mycoltwo{$\alpha^{b}(x,y)=\frac{\pi(y)q(y,x)}{\pi(x)q(x,y)+\pi(y)q(y,x)}$} satisfy \eqref{eq:detailed}. \pause 
\item   For all $\alpha$ satisfying \eqref{eq:detailed}, we have \colbox{$\alpha \leq \alpha^{MH}$}. \mycolthree{To be done on the blackboard.} 
\end{itemize}
\end{frame}

\section{Chap 2: Uniqueness of invariant probability measures}


\begin{frame}
  \frametitle{Uniqueness under irreducibility assumptions}
  \begin{block}{Proposition: \textbf{Irreducible Markov kernels}}
    Assume that there exists a non-null measure \textup{$\mu\in\meas +(\Xset)$}
satisfying the following property:
\begin{enumerate}[$(\star)$]
\item \alert{For all $A\in\Xsigma$ such that $\mu(A)>0$ and for all $x\in\Xset$,
there exists $n\in\nset$ such that $P^{n}(x,A)>0$}.
\end{enumerate}
Then, $P$ admits \colbox{at most one} invariant probability measure.
  \end{block}
  \pause 
  If condition $(\star)$ is satisfied, we say that $P$ is \mycoltwo{$\mu$-irreducible}. \pause 
  \begin{block}{Application: \textbf{Metropolis-Hastings algorithms}}
    Assume that 
    \begin{itemize}
      \item \alert{$Q(x,\rmd y)=q(x,y) \lambda(\rmd y)$ 
      and $\pi(\rmd x)=\pi(x)\lambda(\rmd x)$ with \colbox{$q>0$ and $\pi>0$}.  } 
    \end{itemize}
  
    Then $\mh{\pi,Q}$ admits $\pi$ as its \colbox{unique probability
    measure}.         
  \end{block}
\end{frame}
\begin{frame}
  \frametitle{Proof of the uniqueness of the invariant probability measure for irreducible Markov chains}
  The following lemma is useful for the proof... 
  \begin{block}{Lemma}
    If $P$ admits two distinct invariant
    probability measures, it also admits distinct invariant probability
    measures $\pi_{0}$ and $\pi_{1}$ that are mutually singular, i.e.,
    such that there exists $A\in\Xsigma$ such that $\pi_{0}(A)=\pi_{1}(A^{c})=0$.
  \end{block}
  \pause 
  \mycolthree{To be done on the blackboard.} 
\end{frame}

\section{Chap 3: Dynamical systems}
\begin{frame}
  \frametitle{Dynamical systems}
  \begin{block}{Definition}
    \textbf{(Dynamical systems)}
    A \colbox{dynamical system}\index{dynamical system} $\mathcal{D}$
is a quadruplet $\mathcal{D}=(\Omega,\mcf,\mathbb{P},T)$ where $(\Omega,\mcf,\mathbb{P})$
is a probability space and $T:\Omega\rightarrow\Omega$ is a measurable
mapping such that \fbox{\alert{$\mathbb{P}=\mathbb{P}\circ T^{-1}$}}.
  \end{block}
\pause
  \begin{lemma} \textbf{(Invariant sets)}
    The collection of sets $\mc I=\set{A\in\mcf}{\indi A=\indi A\circ T}$
    is a $\sigma$-field and any set in $\mc I$ is called an \colbox{invariant set.} \index{invariant!set}
    \end{lemma}
    \pause    
    \begin{definition}
      \textbf{(Ergodicity)} \index{ergodic dynamical system} A dynamical system $(\Omega,\mcf,\mathbb{P},T)$
      is said to be \colbox{ergodic} if \colbox{invariant sets are $\PP$-trivial} that is if $A\in\mc I$
      then either $\mathbb{P}(A)=0$ or $\mathbb{P}(A)=1$.
      \end{definition}
      
\end{frame}

\begin{frame}
  \frametitle{The Birkhoff theorem}
  \begin{theorem}
    \textbf{\label{thm:birk:dynam}(The Birkhoff theorem)} \index{Birkhoff's theorem}Let $\mathcal{D}=(\Omega,\mcf,\mathbb{P},T)$
    be an \colbox{ergodic dynamical system} and let $h\in\lfuncset 1(\Omega)$.
    Then,
    \[
    \alert{\lim_{n\to\infty}n^{-1}\sum_{k=0}^{n-1}h\circ T^{k}=\PE[h]\ ,\quad\PP-a.s.}
    \]
    \end{theorem}
    \end{frame}


    
    \section{Chap 3:Markov chains and ergodicity}

\begin{frame}
  \frametitle{}
  Let $S$ be the shift operator: if $\omega=(\omega_k)_{k\in\nset} \in \Xset^{\nset}$, we set $S(\omega)=\omega' \in \Xset^{\nset}$ where $\omega'_k=\omega_{k+1}$ for all $k\in\nset$.  
  \begin{lemma}[MC and dynamical systems]
    Let $P$ be a Markov kernel admitting an \alert{invariant probability measure}
    $\pi$. Then, the quadruplet $(\Xset^{\nset},\Xsigma^{\otimes\nset},\PP_{\pi},S)$
    is a \colbox{dynamical system}.
    \end{lemma}
    \pause
    \begin{theorem}[MC and ergodicity]
      Let $P$ be a Markov kernel on $\Xset\times\Xsigma$. Assume that
      $P$ admits a \alert{unique invariant probability measure} $\pi$. Then, the
      dynamical system $(\Xset^{\nset},\Xsigma^{\otimes\nset},\PP_{\pi},S)$
      is \colbox{ergodic}.
      \end{theorem}
      \pause 
      \mycolthree{The proof of the Theorem will be done on the blackboard. }
\end{frame}  

\begin{frame}
  \frametitle{}
  \begin{theorem}[The Birkhoff theorem for MC]
    Let $P$ be a Markov kernel admitting a
\alert{unique invariant probability measure} $\pi.$ Then, for all $h\in\funcset{}(\Xset^{\nset})$
such that $\PE_{\pi}[|h|]<\infty$, we have
\[
\mycoltwo{\lim_{n\to\infty}n^{-1}\sum_{k=0}^{n-1}h(X_{k:\infty})=\PE_{\pi}[h]\,,\quad\PP_{\pi}-a.s.}
\]
  \end{theorem}
\pause
  \begin{corollary}[LLN Starting from stationarity]
  Let $P$ be a Markov kernel admitting a
  \alert{unique invariant probability measure} $\pi.$ Then, for all $f\in\funcset{}(\Xset)$
  such that $\pi(|f|)=\int_{\Xset}\pi(\rmd x)|f(x)|<\infty$, we have
  \begin{equation}
  \mycoltwo{\lim_{n\to\infty}n^{-1}\sum_{k=0}^{n-1}f(X_{k})=\pi(f)\,,\quad\PP_{\pi}-a.s.\label{eq:markov:LLN}}
  \end{equation}
  \end{corollary}
 
  
 
\end{frame}

\begin{frame}
  \frametitle{}
  \begin{corollary}[Other starting points]
    Let $P$ be a Markov kernel admitting
    a \alert{unique invariant probability measure} $\pi.$ Then, for all $f\in\funcset{}(\Xset)$
    such that $\pi(|f|)=\int_{\Xset}\pi(\rmd x)|f(x)|<\infty$, we have
    \colbox{for $\pi$-almost all $x\in\Xset$},
    \begin{equation}
    \mycoltwo{\lim_{n\to\infty}n^{-1}\sum_{k=0}^{n-1}f(X_{k})=\pi(f)\,,\quad\PP_{x}-a.s.\label{eq:lln:almost}}
    \end{equation}
    \end{corollary}
  \end{frame}

  \begin{frame}
    Assume that \alert{$Q(x,\rmd y)=q(x,y)\lambda(\rmd y)$} and \alert{$\pi(\rmd y)=\pi(y)\lambda(\rmd y)$}
    where \colbox{$q>0$, $\pi>0$} and $\lambda$ is a $\sigma$-finite measure
    on $(\Xset,\Xsigma)$.
  \begin{theorem} \label{thm:lln:hm}
    The Markov chain $\seq{X_{n}}{n\in\nset}$ generated
    by the Metropolis-Hastings algorithm is such that: \alert{for all initial
    distributions $\nu\in\meas 1(\Xset)$} and all $f\in\funcset{}(\Xset)$
    such that $\pi(|f|)=\int_{\Xset}\pi(\rmd x)|f(x)|<\infty$,
    \begin{equation}
    \mycoltwo{\lim_{n\to\infty}n^{-1}\sum_{k=0}^{n-1}f(X_{k})=\pi(f)\,,\quad\PP_{\nu}-a.s\label{eq:LLN:MH}}
    \end{equation}
    \end{theorem}
  
\end{frame}
\begin{frame}
  \frametitle{}
  What if $P$ is not the Markov kernel of a Metropolis-Hastings algorithm? 
  \begin{theorem}
    If $P$ is a Markov kernel on $\Xset \times\Xsigma$ that admits a \colbox{unique invariant} \colbox{probability measure} $\pi$. Assume in addition that for all bounded functions $h$ and all measures $\nu\in\meas 1(\Xset)$,
    \begin{equation}
    \label{eq:setwise}
    \alert{\lim_{n \to \infty}  \nu P^n h=\pi(h)}
    \end{equation}
    \pause 
    Then, \colbox{for all initial
    distributions} $\nu\in\meas 1(\Xset)$ and all $f\in\funcset{}(\Xset)$
    such that $\pi(|f|)=\int_{\Xset}\pi(\rmd x)|f(x)|<\infty$,
    \begin{equation}
    \mycoltwo{\lim_{n\to\infty}n^{-1}\sum_{k=0}^{n-1}f(X_{k})=\pi(f)\,,\quad\PP_{\nu}-a.s\label{eq:LLN:gener}}
    \end{equation}
    \end{theorem}
    
  

\end{frame}

\end{document}
\begin{frame}
\frametitle{Variational Inference within the $\alpha$-divergence family}
\scalebox{0.8}{
\begin{minipage}[t]{1.2\linewidth}


$(\Yset,\Ysigma, \nu)$ : measured space, $\nu$ is a $\sigma$-finite measure on $(\Yset, \Ysigma)$.

$\PQ$ and $\PP$ : $\PQ \preceq \nu$, $\PP \preceq \nu$ with $\frac{\rmd \PQ}{\rmd \nu} = q$, $\frac{\rmd \PP}{\rmd \nu} = p(\cdot|\data)$.

\begin{block}{$\alpha$-divergence between $\PQ$ and $\PP$} \vspace{-0.7cm}
\begin{align*}
\diverg\couple[\PQ][\PP] = \int_\Yset f_\alpha \left(\dfrac{q(y)}{p(y|\data)}\right) p(y|\data) \nu(\rmd y) \eqsp,
\end{align*} \vspace{-0.5cm}

where \vspace{-0.5cm}
\begin{equation*}
  \falpha = \begin{cases}
    \frac{1}{\alpha(\alpha-1)} \left[  u^\alpha -1 - \alpha(u-1) \right], & \mbox{if } \alpha \in \rset \setminus \lrcb{0,1}, \\
    1 - u + u\log(u), & \mbox{if } \alpha = 1 \mbox{ (Forward KL),} \\
    u-1 -\log(u), & \mbox{if } \alpha = 0 \mbox{ (Reverse KL).}
  \end{cases}
\end{equation*} \vspace{-0.3cm}
\end{block} \pause

$\rightarrow$ We can get rid of $p(\data)$ in the optimisation ! \vspace{-0.3cm}
$$
q^\star = \arginf_{q \in \mathcal{Q}} \int_\Yset \falpha \left(\dfrac{q(y)}{p(y)}\right) {p(y)} \nu(\rmd y) \quad \mbox{with } p(y) = p(y, \data) \eqsp.
$$ \vspace{-0.5cm} \pause

$\rightarrow$ $\falpha$ is \mycolthree{convex} !

\end{minipage}}

\end{frame}

\begin{frame}{Our approach}

\scalebox{0.8}{
\begin{minipage}[t]{1.2\linewidth}

\begin{itemize}
  \item Usually in Variational Inference : parametric family \vspace{-0.2cm}
$$
\set{y \mapsto k_\theta(y)}{\theta \in \Tset} \eqsp.
$$ \vspace{-0.7cm} \pause
\item Let us consider a broader approximating family \vspace{-0.2cm}
$$
\set{y \mapsto \int_\Tset \mycoltwo{\mu(\rmd \theta)} k_\theta(y)}{\mu \in \mathsf{M}} \eqsp,
$$
$\mathsf{M}$ : subset of $\meas{1}(\Tset)$, the set of probability measures on $(\Tset, \Tsigma)$. \newline \vspace{-0.4cm}

$\rightsquigarrow$ \mycoltwo{Example} : Mixture models $\mu = \sum_{j=1}^{J} \lambda_j \delta_{\theta_j}$. \vspace{0.4cm} \pause
\end{itemize} \vspace{-0.5cm}

\begin{block}{Optimisation problem}
\begin{itemize}
\item $\mu k(y) = \int_\Tset \mu(\rmd \theta) k(\theta, y)$, where $K:(\theta,A) \mapsto \int_A k(\theta,y)\nu(\rmd y)$ is a  Markov transition kernel on $\Tset \times \Ysigma$ with kernel density $k$
\item $p$ : \mycolthree{measurable positive function} on $(\Yset,\Ysigma)$
\end{itemize}
\begin{center}
\colbox{$
\arginf_{\mu \in \mathsf{M}} \underbrace{\mathlarger{\int_\Yset} \falpha \left(\dfrac{\mu k(y)}{p(y)}\right) {p(y)} \nu(\rmd y)}_{ \eqdef \Psif(\mu)}
$}
\end{center}


\end{block}
\end{minipage}}

\end{frame}


%\begin{frame}
%\tableofcontents
%\end{frame}

%More flexible families of divergence $D$

%$\mathcal{F}$: $f$ divergence

%\begin{itemize}
%  \item $\mathcal{F}$: More flexible families of divergences $D$.
%  \item $q_\theta$: Monte Carlo methods and Black-Box inference techniques.
%  \item Scalable methods (Stochastic Gradient Descent...).
%\end{itemize}





%$\rightarrow$ \mycol{Posterior density approximation} : $p(y) = \posterior$


\section{The \aei descent}

\begin{frame}{The \aei descent}

\scalebox{0.9}{
\begin{minipage}{1.\linewidth}

\underline{Optimisation problem} \vspace{-0.3cm}
$$
\arginf_{\mu \in \mathsf{M}} \Psif(\mu) \quad \mbox{with} \quad \Psif(\mu) \eqdef \mathlarger{\int_\Yset} \falpha \left(\dfrac{\mu k(y)}{p(y)}\right) {p(y)} \nu(\rmd y)
$$


\underline{Algorithm}

Let $\mu_1 \in\meas{1}(\Tset)$ such that $\Psif(\mu_1) < \infty$. We define the sequence of probability measures $(\mu_n)_{n \in \nstar}$ iteratively by
\begin{equation}
\label{eq:def:mu}
\mu_{n+1}= \iteration (\mu_n)\;,\qquad n\in\nstar \eqsp.
\end{equation} \vspace{-0.4cm}

\end{minipage}
}

\begin{minipage}{1.\linewidth}
\begin{center}
\scalebox{0.9}{
\begin{algorithm}[H]
\caption{\em Exact \aei descent one-step transition}
\label{algo:aei}
\noindent \vspace{-0.5cm}
\begin{enumerate}
\item \underline{Expectation step} : \setlength{\parindent}{1cm} $\bmuf(\theta)= \mathlarger{\int_\Yset} k(\theta,y)  \falpha' \left(\dfrac{\mu k(y)}{p(y)}\right) \nu(\rmd y)$
\item \underline{Iteration step} : \setlength{\parindent}{1cm} $\iteration (\mu)(\rmd \theta)= \dfrac{\mu(\rmd \theta) \cdot \GammaAlpha(\bmuf(\theta)+ \cte)}{\mu(\GammaAlpha(\bmuf + \cte))}$
\end{enumerate} \
\end{algorithm}}
\end{center}

\end{minipage} \vspace{0.1cm}

%$\rightarrow$ Example : $\GammaAlpha(v) = e^{-\eta v}$, $\eta > 0$, $\cte \in \Rset$.
%
%\quad \ Infinite-Dimensional \mycolthree{Entropic Mirror Descent}.

\end{frame}

\begin{frame}{Monotonicity}

\begin{flushleft}
\mycoltwo{(A1)} For all $(\theta,y) \in \Tset \times \Yset$, $k(\theta,y)>0$, $p(y)>0$ and $\int_\Yset p(y) \nu(\rmd y)<\infty$. \\ \pause
\end{flushleft}

\begin{flushleft}
\mycoltwo{(A2)} The function $\GammaAlpha: \Domain \to\rset_{> 0}$ is decreasing, continuously differentiable and satisfies the inequality
    \begin{align*}
        \left[(\alpha -1)(v- \cte) + 1\right] (\log {\GammaAlpha})'(v) + 1 \geq 0 \eqsp.
    \end{align*}
\end{flushleft} \pause \vspace{-0.3cm}

\begin{block}{Theorem 1}

  Assume \mycoltwo{(A1)} and \mycoltwo{(A2)}. Let $\mu\in\meas{1}(\Tset)$ be such that $\Psif(\mu)<\infty$ and $\mu(\GammaAlpha(\bmuf + \cte)) < \infty$. Then, the two following assertions hold. \vspace{-0.1cm}
\begin{enumerate}
\item \label{item:mono1} We have  $\Psif \circ \iteration (\mu) \leq \Psif(\mu)$.
\item \label{item:mono2} We have $\Psif \circ \iteration (\mu) =\Psif(\mu)$ if and only if $\mu=\iteration (\mu)$.
\end{enumerate}
\end{block}
\end{frame}


\begin{frame}{Examples satisfying \mycoltwo{(A2)}}

\begin{flushleft}
\mycoltwo{(A2)} The function $\GammaAlpha: \Domain \to\rset_{> 0}$ is decreasing, continuously differentiable and satisfies the inequality
    \begin{align*}
        \left[(\alpha -1)(v- \cte) + 1\right] (\log {\GammaAlpha})'(v) + 1 \geq 0 \eqsp.
    \end{align*}
\end{flushleft} \pause

\begin{enumerate}
  \item Entropic MD : $\eta \in (0,1]$, $\cte \in \rset$ and $\alpha = 1$
  $$
  \GammaAlpha(v) = e^{-\eta v} \eqsp.$$
   \pause
  \item \mycolthree{Power descent :} $\eta \in(0,1]$, $(\alpha-1)\cte \geq 0$ and $\alpha \neq 1$
  $$
  \GammaAlpha(v) =  [ \left(\alpha -1 \right)v + 1]^{\frac{\eta}{1-\alpha}} \eqsp.
  $$

\end{enumerate}
\end{frame}

\begin{frame}{Limiting behavior}
Table 1: Examples of allowed $(\GammaAlpha, \cte)$ in the \aei descent

\includegraphics[scale=0.51]{{table2}.png}
\pause

\begin{tikzpicture}[
          remember picture,
          overlay,
          expl/.style={draw=white,fill=white,rounded corners,text width=.9cm},
          arrow/.style={Orchid!80!black,ultra thick,->,>=latex}
        ]
        %\node<2->[expl] (ak) at (-.9,.8) {$a_K(\theta)$};
        %\node<2-> (akTarget) at (.7,1.3) {};

        %\node<2->[expl] (ck) at (9.5,.8) {};
        %\node<2-> (ckTarget) at (8,1.3) {};

        %\node<3->[expl] (target) at (-0.2,2.7) {\scalebox{0.8}{\color{Orchid!80}{$a_K(\theta) = 0$}} \\ \scalebox{0.8}{$b_K(\theta) = ||\bmufk(\theta)|^\phi-|\bmuf(\theta)|^\phi|$} \\ \scalebox{0.8}{$c_K(\theta) = |\bmufk(\theta)|^\phi + |\bmuf(\theta)|^\phi$}};

        %\draw<2->[arrow] (ak.east) to[out=0,in=200] (akTarget);
        %\draw<2->[arrow] (ck.west) to[out=180,in=340] (ckTarget);


        \draw [line width=0.7mm, Orange] (9.85,0.56) -- (9.85,0.945);

        \draw [line width=0.7mm, Emerald] (9.85,1.0) -- (9.85,2.27);

\end{tikzpicture}

\vspace{-0.2cm}

$\rightarrow$ {\mycolthree{Convergence towards the optimum value at a $O(1/N)$ rate} \vspace{0.2cm}

{$\rightarrow$ {\color{Orange}{Convergence towards the optimum value}} \vspace{-0.2cm}
%$$
%\Psif(\mu^\star) = \inf_{\zeta \in \meas{1, \mu_1}(\Tset)} \Psif(\zeta)
%$$
}


%$$
%\Psif(\mu_N) - \Psif(\mu^\star) \leq \frac{\cteinf}{N} \left[ KL\couple[\mu^\star][\mu_1] + L\frac{ \ctesup}{\ctemono} \delta_1 \right]
%$$
}

\end{frame}

\begin{frame}{An $O(1/N)$ convergence rate?}

\scalebox{0.82}{
\begin{minipage}[t]{1.2\linewidth}

\mycol{\xmark}~$\beta$-smooth assumption \mycol{not satisfied} in general for the $\alpha$-divergence (e.g Gaussian family). \vspace{0.2cm} \pause

\mycoltwo{\cmark} Quantify the improvement in terms of the {\mycoltwo{variance of $\bmuf$}} \vspace{-0.3cm}

\begin{block}{\vspace{-0.3cm}}\vspace{-0.3cm}
$$
\frac{\ctemono}{2} \Var_\mu \left( \bmuf\right) \leq \Psif(\mu) - \Psif\circ \iteration(\mu) \eqsp,
$$\vspace{-0.2cm}

where \scalebox{0.9}{$\ctemono \eqdef \inf_{v \in \Domain} \lrcb{ \left[(\alpha -1) (v- \cte) + 1\right] (\log \GammaAlpha)'(v) + 1 } \times \inf_{v \in \Domain}  - \GammaAlpha'(v)$.}

\end{block} \pause


$\rightsquigarrow$ Projected Gradient descent : $f$ is $\beta$-smooth on $\rset$ \vspace{-0.2cm}
$$
\forall u \in \Rset, \quad \frac{1}{\beta} \| \nabla f(u) \|^2 \leq f(u) - f\left(u- \frac{1}{\beta} \nabla f(u)\right) \eqsp.
$$
\end{minipage}}

\end{frame}

\begin{frame}{Mixture models and \aei descent}

\scalebox{0.8}{
\begin{minipage}[t]{1.2\linewidth}

$S_J = \set{ \lbd{}= (\lambda_1, ..., \lambda_J) \in \rset^J}{ \forall j \in \lrcb{1,..., J}, \eqsp \lambda_j \geq 0 \eqsp \mbox{and} \eqsp \sum_{j=1}^J \lambda_j = 1}$.

Let $\theta_1, ..., \theta_J \in \Tset$ \mycoltwo{be fixed} and denote
\begin{center}
$\mu_{\lbd{}} = \sum_{j=1}^J \lambda_j \delta_{\theta_j} \quad \mbox{with} \quad \lbd{} \in \simplex_J \eqsp.$
\end{center} \vspace{-0.6cm}

\begin{block}{\vspace*{-3ex}}
Then, $\mycoltwo{\mu_n = \underbrace{\iteration \circ \dots \circ \iteration}_{n\ \text{times}}(\mu_{\lbd{}})}$ is of the form
%Initiating by $\mu_0= \mu_{\lbd{}}$, we obtain by induction :
$
\mu_n=\sum_{j=1}^J \lbd[j]{n} \delta_{\theta_j}$
with \vspace{-.5cm}

\begin{center}
\begin{equation}
\begin{cases}
\lbd{1}=\lbd{}  \\
\lbd[j]{n+1}= \frac{\lbd[j]{n} \GammaAlpha(\bmuf[\mu_n](\theta_j) + \cte)}{ \sum_{i=1}^J \lbd[i]{n} \GammaAlpha(\bmuf[\mu_n](\theta_i) + \cte)} \eqsp.
\end{cases}
\end{equation}
\end{center}
\vspace{-0.2cm} \pause
\end{block}

\begin{itemize}
  \item {\mycoltwo{Exploitation step}} which requires no information on the distribution of $\lrcb{\theta_1, ..., \theta_J}$ (as opposed to Importance Sampling) \pause
  \item In practice, we will use
\begin{center}
\scalebox{0.9}{$\bmufk[\mu_n](\theta_j) = \dfrac{1}{M} \sum \limits_{m=1}^M \frac{k(\theta_j, Y_{m, n})}{\mu_n k(Y_{m, n})}  \falpha'\left( \frac{\mu_n k(Y_{m, n})}{p(Y_{m, n})} \right)$,}
\end{center}
with $Y_{1, n}, ..., Y_{M, n}$ drawn independently from $\mu_n k$.
\end{itemize}


%\mycoltwo{Exploitation step}
%  \begin{itemize}
%  \item Free to choose the kernel $K$, the $\alpha$-divergence being optimised.
%
%  \item No information on the distribution of $\lrcb{\theta_1, ..., \theta_J}$ required.
%  \end{itemize}
%  \item {\color{Orchid!80!black}{Exploration}}


%\end{enumerate}
\end{minipage}}

\end{frame}

\section{Numerical Experiments}


\begin{frame}{Numerical Experiments}

\begin{itemize}
\item Framework
\scalebox{0.8}{
\begin{minipage}[t]{1.2\linewidth}
\textbf{Kernel:} Gaussian transition kernel $k_{h_t}$ with bandwidth $h_t$. \vspace{-0.2cm}
$$
\set{y \mapsto \mu_{\lbd{}} k_{h_t}(y) = \sum_{j= 1}^{J_t} \lambda_j k_{h_t}(y- \thetat[j][t])}{\lbd{} \in \simplex_{J_t}, (\thetat[j][t])_{1 \leq j \leq J_{t}} \in \Tset^{J_t} }\eqsp.
$$ \vspace{-0.2cm}

$h_t \propto J_t^{-1/(4 + d)}$, where $d$ is the dimension of the latent space. \newline \vspace{-0.2cm} \pause
\begin{enumerate}
  \item {\mycoltwo{Exploitation step}} Optimise $\lbd{}$ using the Mixture \aei descent (with Monte Carlo approximation of $\bmuf$). \pause
  \item {\color{Orchid!80!black}{Exploration step}} Sample $(\thetat[j][t+1])_{1 \leq j \leq J_{t+1}}$ according to $ \mu_{\lbd{}} k_{h_t}$. \vspace{0.2cm}
\end{enumerate}



%\textbf{Alpha}: $\alpha < 1$. \newline
\end{minipage}} \pause

\item Toy example

\scalebox{0.8}{ $
p(y) = Z \times \left[ 0.5 \mathcal{N}(\boldsymbol{y}; -s \boldsymbol{u_d}, \boldsymbol{I_d}) + 0.5 \mathcal{N}(\boldsymbol{y}; s \boldsymbol{u_d}, \boldsymbol{I_d}) \right]
$} \vspace{0.3cm} \pause

\item Bayesian Logistic Regression \scalebox{0.8}{Covertype dataset ($581,012$ data points and $54$ features)}
\end{itemize}

\end{frame}




\begin{frame}{Toy Example: Mirror Descent vs Power Descent}

\scalebox{0.8}{
\begin{minipage}[t]{1.2\linewidth}
We compare:
\begin{itemize}
\item \underline{$0.5$-Mirror descent :} $\GammaAlpha(v) = e^{-\eta v}$ with $\alpha = 0.5$,
\item \underline{$0.5$-Power descent :} $\GammaAlpha(v) = [ \left(\alpha -1 \right)v + 1]^{\eta/(1-\alpha)}$ with $\alpha =0.5$.
\end{itemize} \vspace{0.2cm}

\quad $J = M = 100$, initial weights: $[1/J, ..., 1/J]$, $N=10$, $T=20$.

\end{minipage}} \vspace{-0.2cm}

\begin{figure}[!ht]
\caption{Average Renyi-Bound for the 0.5-Power and 0.5-Mirror descent computed over 100 replicates with $\eta_0 = 0.5$.}
\label{fig:toy}
\begin{tabular}{ccc}
\includegraphics[scale=0.22]{{mixtureDim8Alpha05}.png}  &
\includegraphics[scale=0.22]{{mixtureDim16Alpha05}.png} & \includegraphics[scale=0.22]{{mixtureDim32Alpha05}.png}
\end{tabular}
\end{figure}
\vspace{-0.5cm} \pause


\scalebox{0.8}{
\begin{minipage}[t]{1.2\linewidth}
\begin{align*}
  \mbox{\color{Mahogany}{Mirror} } \quad \lbd[j]{n} &\propto \exp\left( \frac{\eta}{1-\alpha} \left( (\alpha-1)\bmuf[\mu_{\lbd{n}}](\theta_j) + (\alpha-1)\cte \right) \right) \\
  \mbox{\color{MidnightBlue}{Power} } \quad \lbd[j]{n} &\propto \exp\left( \frac{\eta}{1-\alpha} {\color{MidnightBlue}{\boldsymbol\log} \left(  {\color{Black}{(\alpha-1)\bmuf[\mu_{\lbd{n}}](\theta_j) + (\alpha-1)\cte}} \right)} \right) \eqsp.
\end{align*}
\end{minipage}}

\end{frame}


\begin{frame}{Bayesian Logistic Regression}

\scalebox{0.8}{
\begin{minipage}[t]{1.2\linewidth}

We compare :
\begin{itemize}
\item \underline{$0.5$-Power descent}
\item \underline{AIS} (Adaptive Importance Sampling)
\end{itemize}

$N = 1$, $T = 500$, $J_0 = M_0 = 20$

$J_{t+1} = M_{t+1} = J_t + 1$, initial weights: $[1/J_t, ..., 1/J_t]$.
\end{minipage}}

\begin{figure}[!ht]
\caption{Average Accuracy and Log-likelihood computed over 100 replicates for the 0.5-Power descent ($\eta_0 = 0.05$) and the AIS algorithm.}
\label{fig:compareRealData}
\begin{tabular}{cc}
\includegraphics[scale=0.26]{{covertypeAccLin}.png} &
\includegraphics[scale=0.26]{{covertypeLLHLin}.png}
\end{tabular}
\end{figure}


\end{frame}

\section{Conclusion}

\begin{frame}{Conclusion}

The \aei descent
\begin{itemize}
\item \mycolthree{performs an update of measures}
      \begin{itemize}
        \item sufficient conditions on $(\alpha, \GammaAlpha)$ leading to a systematic {\color{orange}{decrease}}
        \item includes {\color{orange} Entropic Mirror Descent}
        \item convergence to an {\color{orange}optimum} and {\color{orange}$O(1/N)$} convergence rates,
      \end{itemize} \pause \vspace{0.2cm}

\item {\color{Orchid}can be applied to density approximation}
\begin{itemize}
  \item handles the case of {\color{RoyalPurple} Mixture Models} for any kernel $K$
  \item requires no information on the distribution of {\color{RoyalPurple}{$\lrcb{\theta_1, ..., \theta_J}$}}
  \item empirical benefit of using the {\color{RoyalPurple}{Power descent}}. \pause
\end{itemize}
\end{itemize}
\vspace{0.3cm}


%Perspectives : convergence rate for a general $\GammaAlpha$, optimal $\eta$, try other types of Exploration steps, variance reduction ... \newline \pause

\scalebox{0.8}{[1] Kamélia Daudel, Randal Douc and François Portier (2020).}

\scalebox{0.8}{Infinite-dimensional gradient-based descent for alpha-divergence minimisation.}

\scalebox{0.8}{https://hal.telecom-paris.fr/hal-02614605.}

\scalebox{0.8}{[2] Kamélia Daudel, Randal Douc, François Portier and François Roueff (2019).}

\scalebox{0.8}{The $f$-Divergence Expectation Iteration Scheme. https://arxiv.org/abs/1909.12239.}


\end{frame}

\appendix

\begin{frame}

\textbf{Some additional slides...}

\end{frame}

\begin{frame}{Idea of the proof: Monotonicity}
\scalebox{0.8}{
\begin{minipage}[t]{1.2\linewidth}
\textbf{Step 1:} Let $\mu,\zeta\in\meas{1}(\Tset)$ be such that $\zeta \preceq \mu$ and $\Psif(\mu) < \infty$. We show that \vspace{-0.1cm} \begin{center} \colbox{$
A_\alpha \leq  \Psif(\mu) - \Psif(\zeta)$} \end{center} where \vspace{-0.4cm}
\begin{align}\label{eq:Aalpha}
  A_\alpha \eqdef \int_\Yset  \nu(\rmd y)  \int_\Tset \mu(\rmd \theta) k(\theta,y)  \falpha' \left( \frac{g(\theta)\mu k(y) }{p(y)} \right) \left[ 1 - g(\theta) \right]
\end{align}
and $\zeta(\rmd \theta)=\mu(\rmd \theta) g(\theta)$. Furthermore equality holds iif $\zeta=\mu$. \vspace{0.5cm}
%
%\begin{block}{\vspace*{-3ex}} \vspace{-0.2cm}
%$$
%h_\alpha(\zeta, \mu) = \int_\Yset \nu(\rmd y) p(y)  \int_\Tset \frac{\mu(\rmd \theta) k(\theta,y)}{\mu k(y)} \falpha \left( \frac{g(\theta)\mu k(y)}{p(y)} \right)
%$$
%\end{block} \pause

\textbf{Step 2:} We set $g \propto \GammaAlpha(\bmuf + \cte)$ in \eqref{eq:Aalpha} and we show that $A_\alpha \geq 0$. \vspace{-0.3cm}

\begin{block}{\vspace*{-3ex}}
Probability space $(\Tset,\Tsigma,\mu)$, $V(\theta)=\bmuf(\theta)+ \cte$ and $\tgamma (v) = \GammaAlpha(v) / \mu(\GammaAlpha(\bmuf + \cte))$\vspace{-0.3cm}
\begin{align*}
\alpha \in \rset \setminus \lrcb{1} : \quad &A_\alpha = \Cov \left( \left(V - \cte + \frac{1}{\alpha-1} \right)\tgamma^{\alpha-1}(V), 1- \tgamma(V) \right) \\
&A_1 = \Cov(\log \tgamma (V) + V, 1- \tgamma(V)) \eqsp.
\end{align*}
\end{block}

\end{minipage}}
\end{frame}




\begin{frame}{Limiting behavior (1)}
\vspace{-0.2cm}
\scalebox{0.78}{
\begin{minipage}[t]{1.2\linewidth}
 \mycoltwo{(A3)} The function $\GammaAlpha: \Domain \to\rset_{> 0}$ is $L$-smooth and the function $- \log \GammaAlpha$ is concave increasing.
\vspace{-0.9cm}

\begin{align*}
\cteinf^{-1} &\eqdef \inf_{v \in \Domain} (- \log \GammaAlpha)' (v) \eqsp, \\
\ctesup^{-1} &\eqdef \inf_{v \in \Domain} \GammaAlpha(v) \eqsp, %\\
\end{align*}
\vspace{-0.6cm}

{$\meas{1, \mu_1}(\Tset)$ : set of probability measures dominated by $\mu_1$. \vspace{-0.1cm}}


\begin{block}{Theorem 2}
Assume \mycoltwo{(A1)}, \mycoltwo{(A2)}  and \mycoltwo{(A3)}. Further assume that $\ctemono$, $\cteinf > 0$ and that $0 <\inf_{v \in \Domain} \GammaAlpha(v) \leq \sup_{v \in \Domain} \GammaAlpha(v) < \infty$. Moreover, let $\mu_1\in\meas{1}(\Tset)$ be such that
$\Psif(\mu_1)<\infty$. Then, the following assertions hold.

\begin{enumerate}
\item \label{item:admiss1} The sequence $(\mu_n)_{n\in\nstar}$ defined by \eqref{eq:def:mu} is well-defined and the sequence
$(\Psif (\mu_n))_{n\in\nstar}$ is non-increasing.

\item \label{item:admiss2} For all $N \in \nstar$, we have
\begin{align}\label{eq:rate}
  \frac{1}{N} \sum_{n=1}^{N} \Psif(\mu_n) - \Psif(\mu^\star) \leq \mycol{\frac{{\color{black}{\cteinf}}}{N}} \left[ KL\couple[\mu^\star][\mu_1] + L\frac{ \ctesup}{\ctemono} \delta_1 \right]\eqsp,
\end{align}
where $\mu^\star$ is such that $\Psif(\mu^\star) = \inf_{\zeta \in \meas{1, \mu_1}(\Tset)} \Psif(\zeta)$ and where we have defined $\delta_1 = \Psif(\mu_1) - \Psif(\mu^\star)$ and $KL\couple[\mu^\star][\mu_1] = \int_\Tset \log \left(\frac{\rmd\mu^\star}{\rmd\mu_1} \right) \rmd\mu^\star$.
\end{enumerate} \vspace{-0.1cm}
\end{block}
\end{minipage}}


\end{frame}

\begin{frame}{Limiting behavior (2)}

\scalebox{0.82}{
\begin{minipage}[t]{1.2\linewidth}
\mycoltwo{(A4)}
\begin{enumerate}
\item \label{item:theta:one}$\Tset$ is a compact metric space and $\Tsigma$ is
  the associated Borel $\sigma$-field;
\item \label{item:theta:two} for all $y \in \Yset$, $\theta \mapsto k(\theta,y)$ is continuous;
\item \label{item:theta:four} we have $\displaystyle\int_\Yset {\sup_{\theta \in \Tset}k(\theta,y)} \times \sup_{\theta' \in \Tset} \lr{\frac{k(\theta',y)}{p(y)}}^{\alpha-1} \nu(\rmd y)<\infty$.
\end{enumerate}

\begin{block}{Theorem 3}
  Assume \mycoltwo{(A1)} and
  \mycoltwo{(A4)}. Let $\alpha < 1$, $\cte \leq 0$ and set $\GammaAlpha(v) = [ \left(\alpha -1 \right)v + 1]^{\eta/(1-\alpha)}$ for all $v \in \Domain$. Then, for all
  $\zeta\in\meas{1}(\Tset)$, any $\eta > 0$ satisfies $\zeta(\GammaAlpha(\bmuf[\zeta] + \cte)) < \infty$ and $\Psif(\zeta) < \infty$.

  Let $\eta \in (0,1]$. Further assume that there exists $\mu_1,\muf \in\meas{1}(\Tset)$ such that the (well-defined) sequence $(\mu_n)_{n\in\nstar}$ defined by \eqref{eq:def:mu} weakly converges to $\muf$ as
  $n\to\infty$. Then the following assertions hold
\begin{enumerate}
%\item $\phi$ is $\muf$-admissible,
\item $(\Psif(\mu_n))_{n\in\nstar}$ is non-increasing,
\item $\muf$ is a fixed point of $\iteration$,
\item $\Psif(\muf)=\inf_{\zeta \in \meas{1,\mu_1}(\Tset)} \Psif(\zeta)$.
\end{enumerate}
\end{block}
\end{minipage}}

\end{frame}

\begin{frame}{Mixture Stochastic \aei descent}

\scalebox{0.82}{
\begin{minipage}[t]{1.2\linewidth}
\begin{algorithm}[H]
\caption{{\em Mixture Stochastic \aei descent}}
\label{algo:mixture}
\textbf{Input:} $p$: measurable positive function, $K$: Markov transition kernel, $M$: number of samples, $\Theta_J = \{\theta_1, ..., \theta_J\} \subset \Tset$: parameter set. \\
\textbf{Output:} Optimised weights $\lbd{}$. \\
\BlankLine
Set $\lbd{} = [\lambda_{1, 1}, ..., \lambda_{J,1}]$.\\
\While{not converged}{
\begin{enumerate}
\item \underline{Sampling step} : \setlength{\parindent}{1cm} Draw independently $M$ samples $Y_{1}, ..., Y_{M}$ from $\mu_{\lbd{}} k$.
\item \underline{Expectation step} : \setlength{\parindent}{1cm} Compute $\boldsymbol{B}_{\lbd{}} = (b_{j})_{1 \leq j \leq J}$ where \vspace{-0.2cm}
\begin{align}\label{eq:aj}
b_{j} = \dfrac{1}{M} \sum \limits_{m=1}^M \frac{k(\theta_j, Y_{m})}{\mu_{\lbd{}} k(Y_{m})}  \falpha'\left( \frac{\mu_{\lbd{}} k(Y_{m})}{p(Y_{m})} \right)
\end{align}
and deduce $\boldsymbol{W}_{\lbd{}}  = (\lambda_j \GammaAlpha(b_{j} + \cte))_{1\leq j\leq J}$ and $w_{\lbd{}}  = \sum_{j=1}^J \lambda_j \GammaAlpha(b_{j} + \cte)$.
\item \underline{Iteration step} : \setlength{\parindent}{1cm} Set \vspace{-0.2cm}
\begin{align*}
\lbd{} \leftarrow \frac{1}{w_{\lbd{}} } \boldsymbol{W}_{\lbd{}}
\end{align*} \vspace{-0.2cm}
\end{enumerate}
}
\end{algorithm}
\end{minipage}}

\end{frame}

\begin{frame}{Renyi-bound [3]}

\scalebox{0.8}{
\begin{minipage}[t]{1.2\linewidth}
For any variational density $q$, we have
\begin{align*}
\mathcal{L}_{1}(q; \data) & \eqdef \int_\Yset \log \left( \frac{p(y, \data)}{q(y)} \right) q(y) \nu(\rmd y ) \\
\mathcal{L}_\alpha(q; \data) & \eqdef \frac{1}{1-\alpha} \log \left( \int_\Yset \left( \frac{p(y, \data)}{q(y)} \right)^{1-\alpha} q(y) \nu(\rmd y ) \right) \eqsp,
\end{align*}
and in our case
\begin{align}\label{eq:renyibound}
\mathcal{L}_{1}(\mu_n k, \data) &= -\sum_{j=1}^J \lbd[j]{n} \bmuf[\mu_n](\theta_j) \\
\mathcal{L}_\alpha(\mu_n k, \data) &= \frac{1}{1-\alpha} \log \left( (\alpha-1)\sum_{j=1}^J \lbd[j]{n} \bmuf[\mu_n](\theta_j) + 1 \right) \eqsp. \nonumber
\end{align}

[3] Yingzhen Li and Richard E Turner. R\'enyi divergence variational inference. In D. D.
Lee, M. Sugiyama, U. V. Luxburg, I. Guyon, and R. Garnett, editors, Advances in
Neural Information Processing Systems 29, pages 1073–1081. Curran Associates, Inc.,
2016.
\end{minipage}}

\end{frame}


\begin{frame}{Toy Example}

\scalebox{0.8}{
\begin{minipage}[t]{1.2\linewidth}
We compare:
\begin{itemize}
\item \underline{$1$-Mirror descent :} $\GammaAlpha(v) = e^{-\eta v}$ with $\alpha = 1$,
\item \underline{$0.5$-Power descent :} $\GammaAlpha(v) = [ \left(\alpha -1 \right)v + 1]^{\eta/(1-\alpha)}$ with $\alpha =0.5$.
\end{itemize}
\end{minipage}}

\begin{figure}[!ht]
\caption{Plotted is the average Log-likelihood for 0.5-Power and 1-Mirror descent in dimension $d = \lrcb{8, 16,32}$ computed over 100 replicates with $\eta_0 = 0.5$.}
\label{fig:compare}
\begin{tabular}{ccc}
\includegraphics[scale=0.24]{{mixtureDim8llh}.png} &
\includegraphics[scale=0.24]{{mixtureDim16llh}.png} &
\includegraphics[scale=0.24]{{mixtureDim32llh}.png}  \\
\end{tabular}
\end{figure}


\end{frame}

\begin{frame}{Bayesian Logistic Regression}

Same setting as in [4]:

Inference on $p(y|\data)$ with $y = [w, \log \beta]$, $c = \lrcb{-1, 1}$, $x \in \Rset^d$.

\begin{equation*}
\begin{cases}
  p_0(w|\beta) = \mathcal{N} (w, \beta^{-1}
) \\
  p_0(\beta) = \mathrm{Gamma}(\beta, 1, 0.01) \\
  p(c=1|x, w) = \frac{1}{1+e^{-w^T x}}
\end{cases}
\end{equation*} \vspace{0.3cm}

\scalebox{0.8}{[4] Samuel Gershman, Matt Hoffman, and David Blei.}

\scalebox{0.8}{Nonparametric variational inference. In Proceedings of the 29th International }

\scalebox{0.8}{Conference on Machine Learning,
Edinburgh, Scotland, UK, 2012.}

\end{frame}

\begin{frame}{Mass-covering/Mode-seeking}
%
%\begin{align*}
%\lbd[j]{n+1}= \frac{\lbd[j]{n} \GammaAlpha(\bmufk[\mu_n](\theta_j) + \cte)}{ \sum_{i=1}^J \lbd[i]{n} \GammaAlpha(\bmufk[\mu_n](\theta_i) + \cte)} \eqsp,
%\end{align*}
%with $Y_{1, n}, ..., Y_{M, n}$ drawn independently from $\mu_n k$ and
%\begin{align*}
%\bmufk[\mu_n](\theta_j) = \dfrac{1}{M} \sum \limits_{m=1}^M \frac{k(\theta_j, Y_{m, n})}{\mu_n k(Y_{m, n})}  \falpha'\left( \frac{\mu_n k(Y_{m, n})}{p(Y_{m, n})} \right) \eqsp.
%\end{align*}

\scalebox{0.8}{
\begin{minipage}[t]{1.2\linewidth}
\begin{itemize}
\item \underline{Mirror descent :} $\GammaAlpha(v) = e^{-\eta v}$ with $\alpha = 1$. Then,
    \begin{align*}
    \bmufk[\mu_n](\theta_j) &= \dfrac{1}{M} \sum \limits_{m=1}^M \frac{k(\theta_j, Y_{m, n})}{\mu_n k(Y_{m, n})}  \log\left( \frac{\mu_n k(Y_{m, n})}{p(Y_{m, n})} \right)
    \end{align*}
    and
    \begin{align*}
      \lbd[j]{n+1} \propto \prod \limits_{m=1}^M \left( \frac{p(Y_{m, n})}{\mu_n k(Y_{m, n})} \right)^{\frac{\eta}{M}\frac{k(\theta_j, Y_{m, n})}{\mu_n k(Y_{m, n})}} \eqsp.
      \end{align*}%

\item \underline{Power descent:} $\GammaAlpha(v) = ((\alpha-1)v + 1)^{\eta/(1-\alpha)}$. Then,
    \begin{align*}
    \bmufk[\mu_n](\theta_j) &= \dfrac{1}{M(\alpha-1)} \sum \limits_{m=1}^M \frac{k(\theta_j, Y_{m, n})}{\mu_n k(Y_{m, n})}  \left( \frac{\mu_n k(Y_{m, n})}{p(Y_{m, n})} \right)^{\alpha-1} - \frac{1}{\alpha-1}
    \end{align*}
    and
    \begin{align*}
      \lbd[j]{n+1} \propto \left( \frac{1}{M} \sum \limits_{m=1}^M \frac{k(\theta_j, Y_{m, n})}{\mu_n k(Y_{m, n})}  \left( \frac{p(Y_{m, n})}{\mu_n k(Y_{m, n})} \right)^{1-\alpha} + \cte(\alpha-1) \right)^{\frac{\eta}{1-\alpha}} \eqsp.
      \end{align*}

\end{itemize}
\end{minipage} }



\end{frame}



\end{document} 